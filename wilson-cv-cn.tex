\documentclass[11pt,a4paper]{moderncv}

% moderncv themes
%\moderncvtheme[blue]{casual}                 % optional argument are 'blue' (default), 'orange', 'red', 'green', 'grey' and 'roman' (for roman fonts, instead of sans serif fonts)
\moderncvtheme[blue]{classic}                % idem
\usepackage{xunicode, xltxtra}
\XeTeXlinebreaklocale "zh"
\widowpenalty=10000

%\setmainfont[Mapping=tex-text]{文泉驿正黑}

% character encoding
%\usepackage[utf8]{inputenc}                   % replace by the encoding you are using
\usepackage{CJKutf8}

% adjust the page margins
\usepackage[scale=0.88]{geometry}
\recomputelengths                             % required when changes are made to page layout lengths
\setmainfont[Mapping=tex-text]{Times New Roman}         %Times New Roman  Arial
\setsansfont[Mapping=tex-text]{微软雅黑}
\CJKtilde

% personal data
\firstname{王保生}
\familyname{}
\title{}               % optional, remove the line if not wanted

\mobile{+86 18721962910}                    % optional, remove the line if not wanted
\email{wilson\_91@foxmail.com}                      % optional, remove the line if not wanted
%% \quote{\small{``Do what you fear, and the death of fear is certain.''\\-- Anthony Robbins}}

\nopagenumbers{}

\begin{document}

\maketitle

\section{教育背景}
\cventry{2013至今}{硕士}{同济大学计算机系}{将于 2016 年3月毕业} {}{} % 第3到第6 编码可留白
\cventry{2009--2013}{本科}{兰州交通大学计算机系}{1/127}{}{}{}

%\section{毕业论文}
%\cvitem{题目}{\emph{题目}}
%\cvitem{导师}{导师}
%\cvitem{说明}{\small 论文简介}

\section{社区}
\cventry{Blog}{\url{https://king1991wbs.github.com/wilson}}{}{}{}{}
\cventry{StackOverflow}{\url{https://stackoverflow.com/users/3322579/wilson}}{}{}{}{}
\cventry{GitHub}{\url{https://github.com/king1991wbs}}{}{}{}{}

\section{项目经历}
\renewcommand{\baselinestretch}{0.8}

\cventry{2014.06~2014.11}
{机械臂焊接}
{C++, PYTHON}
{创业项目}{}
{实验室项目,希望研发一套机器,应用计算机视觉的相关知识利用摄像头自动识别焊缝并对焊缝进行坐标定位,指挥机械臂进行焊接,并在焊接过程中可以做到自动焊缝纠偏。为了实验和操作的便捷性,开发一套iOS app 和WEB 平台来通过手持设备灵活对机器进行操作。本人主要工作是负责焊缝识别算法的研究与实现。}

\vspace*{0.2\baselineskip}
\cventry{2013.10~2014.07}
{敦煌壁画拼接}{}
{研究项目}{}
{敦煌壁画是国家宝贵的艺术遗产,保护和保存敦煌壁画是必要工作。敦煌研究院拍摄了大量壁画图像,我们的工作就是利用图像拼接技术将这些大尺度高质量图像以窟(敦煌壁画分布在不同的石窟当中)为单位拼接起来。本人主要工作是调试开源算法,比较效率与效果。}

\section{实习经历}
\cventry{2015.06--2015.09}{微软}{Satori}{WEB API,RESTful,C\#,SCOPE}{}
{Satori是Bing的后台小组之一,主要负责Bing的数据支持工作,我所在的小组主要负责网页数据的抽取、分析和整理。在实习期间主要做了三件事情。1.实现Suggestion Service(SS):根据不同的Url和请求从不同的数据源中提取所需信息,整理返回给用户。2.WrapStarAPI性能优化:SS提供了请求数据和插入数据两个接口,但请求数据服务耗时过长。通过分析后发现导致性能瓶颈的主要原因是所使用的WrapStarAPI库,分析修改其代码后,降低了API调用所需时间,从而提升了SS的服务性能。
	3.数据处理:Cosmos是微软的大数据处理平台,通过SCOPE脚本来操作。利用这个工具,实习期间做了一些简单的大数据处理工作。}

\cventry{2015.01--2015.03}{亮风台}{实习算法工程师}{C/C++,OPENCV}{}
{“应用拍拍”是HiScene开发的一款手机APP。本人的工作是参与软件开发过程中所需图像算法的研究与实现。除此之外,负责图像OCR软件的DEMO开发工作。}

%\cvitem{HiScene}{item description}

\section{奖项}
\cventry{2013 }{保送同济研究生}{}{}{}{}
\cventry{2012 }{万侨奖学金 校二等奖学金}{}{}{}{}
\cventry{2009-2011}{国家励志奖学金 校一等奖学金}{}{}{}{}

\section{技能}
\cventry{语言}{C++}{C\#}{}{}{}
\cventry{工具}{GCC}{GDB, OPENCV, TEX}{}{}{}
%\cventry{内核}{Linux, Xen, Android}{}{}{}{}
%% \cventry{应用开发}{Ruby on Rails, iOS}{}{}{}{}
\cventry{平台}{windows}{linux(fedora/ubuntu)}{}{}{}
\cventry{英语}{CET-6}{流利的英语读写能力}{}{}{}

%\section{个人兴趣}
%\cvitem{体育:}{慢跑,小球(羽毛球,乒乓球等)}
%\cvitem{生活:}{吃货一枚,行赏厨艺尽管厨艺不精}
%\cvitem{小清新:}{音乐,电影,读书,素描小白正在入门}


% \cvline{Photography}{\small Digital photography is my newest hobby.}

\closesection{}                   % needed to renewcommands
\renewcommand{\listitemsymbol}{-} % change the symbol for lists

\end{document}
